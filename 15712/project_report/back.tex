\section{Introduction}

Traditionally, circuit switching and packet switching have been
considered as alternative design choices, each with their own advantages
and disadvantages. Packet switches are efficient at multiplexing traffic
across a large number of ports, while circuit switches can often service
higher line rates in a more cost-effective manner, especially when
combined with optical links. However, as optical circuit switching
technology advances, the reconfiguration time (i.e., the amount of time
it takes to alter the input-to-output mapping) is swiftly decreasing,
thus increasingly blurring the division between the packet and circuit
regimes.

One key domain in which these design tradeoffs are being revisited today
is datacenter switching. The datacenter environment, aggregating
tremendous amounts of compute and storage capacity, has driven demand
for ever increasing port counts and line speeds. However, supporting
these demands with existing packet switching technology is becoming
increasingly expensive—in cost, heat and power. Moreover, studies have
shown that the traffic demand inside a datacenter (both within and
across racks) is frequently highly nonuniform, with a large fraction of
the traffic at each switch port destined for a small number of output
ports~\cite{Kandula:2009-2}.

Based on these observations, researchers proposed hybrid datacenter
network architectures that offer higher throughputs at lower cost by
combining switching technologies. In particular, recent proposals
suggest employing highspeed optical~\cite{Chen:2012, Farrington:2010,
Wang:2010} or wireless~\cite{Halperin:2011, Kandula:2009, Zhou:2012}
networks configured to service the heavy flows, while passing the
remainder of the traffic through a traditional, relatively
underprovisioned packet-switched network. 

As technology trends usher in dramatically faster reconfiguration times,
the distinction between packet and circuit is blurred, and ever smaller
flows can take advantage of a hybrid fabric. This trend will soon allow
servicing the bulk of the traffic through a rapidly reconfigurable
optical switch~\cite{Porter:2013}, leaving a relatively minor portion to
be serviced by the packet network~\cite{Liu:2014}. While the potential
cost savings that hybrid technologies could realize is large, the design
space for scheduling resources in the hybrid regime is not yet well
understood. 

Many of the classical approaches to scheduling for switches with
non-trivial reconfiguration delays divide the offered demand into two
parts: an initial, heavy-weight component that is served by O(N) highly
utilized configurations with significant durations, and a second,
residual component that is serviced by a similar number of short,
under-utilized schedules. Recent Solstice scheduling
algorithm~\cite{Liu:2014} exploits the skewed nature of datacenter
traffic patterns to create a small number of configurations with long
durations that minimize the penalty for reconfiguration and leaves only
a small amount of residual demand to be serviced by a low-speed (and
lowcost) unconstrained packet switch.

Since the 

To ensure high overall circuit utilization, each circuit configuration
must be maintained for a relatively long period with respect to the
reconfiguration delay. Since the configuration time is still
non-trivial, it is still hard to schedule small traffic demands
efficiently on optical switch. Scheduling algorithms are forced to
serve these small demands by packet switch, which is slower and less
cost-effective.

To serve these small traffic demands on optical switches efficiently, we
propose an indirection scheduling technique: instead of scheduling an
extra link to serve the traffic, we could leverage existing links to
indirectly serve the traffic. In this way, small traffic demands can
be served by optical switch without additional configuration time.

We first evaluate the indirection technique by formulating the scheduling
problem into an integer linear programming (ILP) problem. Using ILP solver
to compare the optimal solution with and without indirection, we show that
using indirection could reduce the number of configurations and the
total transmission time.

We then evaluate the indirection technique by simulations. We implemented a
preliminary indirection heuristic and applied it to the Solstice algorithm.

The rest of this report is organized as follows.
Section~\ref{sec:indirection} described the scheduling problem and the
proposed indirection technique. Section~\ref{sec:ilp} described the ILP
formulation and the solver results. Section~\ref{sec:simres} describes
the preliminary indirection heuristic and the simulation results.
Finally, we summarize this project in Section~\ref{sec:conclusion}.

\section{The Indirection Heuristic}
\label{sec:indirection}
In this proposoal, we propose to improve the Solstice scheduling
algorithm by introducing a indirection heuristic that reduce the number
of configuration needed for the scheduling. Originally, Solstice will
schedule each traffic demand by a direct path, which means that a new
configuration is required to fulfill a demand that has different sources
or destinations. Our heuristic proposes to schedule the small traffic
demands indirectly. For example, port a has 5 demand to port b and 20
demand to port c; port c has 20 demand to port b. Solstice will require
3 configurations for these demands. However, we could let port a send
all 25 demand to port c and let port c send 25 demand to port b. In this
way we reduce one configuration time, which is nontrivial for optical
switches. 

\section{ILP Formulation of the Scheduling Problem}
\label{sec:ilp}
